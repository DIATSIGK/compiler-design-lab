\documentclass[a4paper, 12pt]{report}

\begin{document}
    \title{Compiler Design Lab Report}
    \author{
        \Large{\textbf{Santhisenan A}} \\
        University Roll Number \textit{TVE15CS050} \\ 
        Department of Computer Science \\
        College of Engineering Trivandrum \\}
    \date{Aug - Nov 2018}
    \maketitle
    \newpage

    \chapter{Automata Concepts}

    \section{Finding $\epsilon$ - closures }
    \subsection{Problem}
    Write program to find $\epsilon$ – closure of all states of 
    any given NFA with $\epsilon$ transition
    \subsection{Algorithm}
    The algorithm for finding $\epsilon$ Closure is as follows:
    \begin{itemize}
        \item{Iterate through all the states in the NFA}
        \item{For each state, find the states reachable via epsilon transitions
        and add them to the epsilon closure.}
        \item{For the all the states added to epsilon closure, recursively 
        execute the function to find the epsilon closure.}
    \end{itemize}

    \subsection{Code}
    \begin{verbatim}
void find_e_closure(nfa n, int state, unordered_set<int> &closure) {
    unordered_set<int> toStates = n.table[state][0];
    closure.insert(state);
    if (toStates.find(-1) != toStates.end())
    {
        return;
    }
    else
    {
        unordered_set<int>::iterator itr;
        for (itr = toStates.begin(); itr != toStates.end(); itr++)
        {
            if (find(closure.begin(), closure.end(), *itr) == closure.end())
            {
                closure.insert(*itr);
            }
            find_e_closure(n, *itr, closure);
        }
    }
}
    \end{verbatim}
    \newpage
    
    \section{Conversion from an NFA with $\epsilon$ transitions to an NFA 
    without $\epsilon$ transitions}
    
    \subsection{Problem}
    Write program to convert an NFA with $\epsilon$ transitions to an NFA 
    without $\epsilon$ transitions.
    \section{Algorithm}
    \begin{itemize}
        \item{Initialize an empty object of type nf a with variable name t}
        \item{Initialize t.num states = a.num states, t.num alphabets = a.num alphabets
and t.final states = a.final states}
        \item{Iterate through each of the state i in Q
        \begin{itemize}
            \item {Initialize l to the ε closure of state i of ε-NFA a – Iterate through each of the input symbol j in Σ}
                \begin{itemize}
                    \item {Initialize an empty list of states f ∗ Iterate through each state k in l and Add all states of a.transition table[k][j + 1] to f ∗ Remove all the duplicates from f}
                    \item {Compute the ε-closure c of f}
                    \item {Set t.transition table[i][j] = c}
                \end{itemize}
        \end{itemize}}
• \item{Return t as the NFA without ε-transitions corresponding to the ε-NFA a}
    \end{itemize}

    \subsection{Code} 
    \begin{verbatim}
        // To convert e-NFA to NFA
void enfa_to_nfa(nfa n, nfa &m)
{
    m.states = n.states;
    m.alphabets = n.alphabets;

    int s = n.states, a = n.alphabets;

    for (int i = 0; i < s; i++)
    {
        vector<unordered_set<int> > row;

        // Insert -1 to alphabet epsilon
        unordered_set<int> nullState;
        nullState.insert(-1);
        row.push_back(nullState);
        nullState.clear();

        // Find the e Closure of the current state and store in eClosure
        unordered_set<int> eClosure;
        unordered_set<int>::iterator itr;
        find_e_closure(n, i, eClosure);

        for (int j = 1; j < a; j++)
        {
            unordered_set<int> temp, tempClosure, toStates;
            unordered_set<int>::iterator it;
            for (itr = eClosure.begin(); itr != eClosure.end(); itr++)
            {
                temp = n.table[*itr][j];
                if (!temp.empty())
                {
                    for (it = temp.begin(); it != temp.end(); it++)
                    {
                        // cout << "Hello" << *it << endl;
                        if(*it != -1) {
                            find_e_closure(n, *it, tempClosure);
                            append_sets(toStates, tempClosure);
                        }
                    }
                }
            }
            if(toStates.empty()) {
                toStates.insert(-1);
            }
            row.push_back(toStates);
            toStates.clear();
        }
        m.table.push_back(row);
        row.clear();
    }
}
    \end{verbatim}

    \newpage
    
    \section{Conversion from an NFA to a DFA}
    
    \subsection{Problem}
    Write program to convert an NFA with $\epsilon$ transitions to an NFA 
    without $\epsilon$ transitions.
    \section{Algorithm}
    \begin{itemize}
        \item{Start with the start of the DFA}
        \item {Find the epsilon closure of the start state.}
        \item {For each input symbol, find the states to which the NFA goes.}
        \item {Group all those states together and name it as a new state in the nfa.}
        \item {Perform the above steps till all the states have been traversed.}
    \end{itemize}

    \subsection{Code} 
    \begin{verbatim}
        // To convert e-NFA to NFA
void nfa_to_dfa(nfa n, dfa &m)
{
    m.alphabets = n.alphabets;
    for(int i = 0; i < n.alphabets; i++) {
        vector <bitset <10> > DFARow;
        for(int j = 0; j < n.states; j++) {
            unordered_set <int> eClosure;
            find_e_closure(n, j, eClosure);
            bitset <10> toStatesDFA(0);
            for(int k = 0; k < eClosure.size(); k++) {
                unordered_set <int> toStates;
                toStates = n.table[i][k];
                unordered_set <int> :: iterator itr;
                for(itr = toStates.begin(); itr != toStates.end(); itr++) {
                    if(*itr != -1) {
                        toStatesDFA.set(*itr);
                    }
                }
            }
            DFARow.push_back(toStatesDFA);
        }
        m.table.push_back(DFARow);
    }
}
    \end{verbatim}
\end{document}
